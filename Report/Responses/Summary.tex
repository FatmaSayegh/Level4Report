\section{Feedback on UI/UX}
Fluid, single-page application feel, 
the visuals and animation were pleasant. easy to navigate.
Particularly liked the treewidth animation.
Maybe the topic words could be clickable to make navigation easier.
Well done! Easy-to-use interface! There are occasional spelling errors that
   could be corrected :)



\section{Feedback on Best Explained}
This is the first time I managed to understand tree width! I liked the
way it was broken into steps.

colouring

Colouring, tree width.

I thought that max k cut was very well explained - the use of cutlines is rather nice.

"Treewidth due to identification of "blocks""

Graph isomorphism and vertex cover - the animations were great, but
each it also had an interactive tasks"""

Tree Width is a hard problem to understand , you
have chosen an extraordinary way in visualising and explaining this
problem, made it seem simple and easy to learn. 
Max Cut , I love the way
the graph transforms then you add the cutlines it was creative. Graph
colouring it is nice that you are able to colour the graph and see the
edges get highlighted when a mistake has been made.  """

Tree Width ,
transforms to a "tree look alike " , the example you chose to visualise
was very simple, and first step to learning it . Yes it is a
"Visualisation web app" serves it purpose """

Tree width and Graph Isomorphism . I liked the pieced and decomposition .
I finally managed to start understanding Tree Width!  Graph Isomorphism
the game was nicely done.

Treewidth. This is not a concept taught in the standard first course on
algorithms for undergrads, so well done!

\section{Feedback on not effective to be understood}
I think all were effective, but it might have been nice to see animations 
and examples for graph colouring before doing the exercise.

Tree width. The application didn't full explain tree width, it just moved the
graph a bit. I think tree width is just on a different level compared to the
other graph theory concepts, and so it would need more detail dedicated to it.
Perhaps it could be marked as an advanced topic.

Perhaps isomorphism. Perhaps it a bit more instructive, if possible?

Treewidth wasn't very clear, unfortunately - I think it wasn't clear that the
treewidth of a graph is the smallest maximum bag size - 1 across all tree
decompositions. So it seems like a graph can have multiple different
treewidths, when it actually has multiple possible widths based on tree
decompositions, then a treewidth that doesn't depend on which tree
decomposition you use. It would have maybe been nice to include some detail on
how treewidth is actually computed, though this is maybe a bit out of scope.

Graph isomorphism - maybe highlight which vertices in source are mapped to
those in target

This is very minor but the max cut animation could have added some way to
highlight the different groups of vertices.

Not at all , all of them are very effective and important . If i had to comment
i would probably say even though i did say Tree Width was my favourite maybe
you can try choosing better choice of words in the "explanation panel " such as
"possible width decomposition" . 


\section{Overall feedback}
* I really liked the animations, and especially the places where you hover over vertices and the edges light up
  * When doing the exercise for isomorphisms, it might be nice to do a similar "edge hover", because the exercise was more difficult without it
  * A lot of the time, "next animation / next task" cycled through rather than being greyed out, so it wasn't easy to tell when the task / activity had finished
  * Maybe it'd be good to have an example of graph colouring before going into the exercise?
  * For tree width: are all graphs representable in that cellular structure?
  * Small typo: funcionality instead of functionality

Overall: lovely app! Really liked it!

Bit of an unfortunate typo on the about page... I assume you don't want to say
that Sofiat's guidance was "detrimental" in building the app! On the whole, I
really like the app! The design and presentation is clear and colourful, and I
think it's a really good basis for giving an intuition of different concepts in
graph theory. Some nice possible extensions could be to allow for custom user
graphs, or provide a bit more detail on algorithms to compute these different
things like vertex covers.


Very nice application!

On the whole, I really like the app! The design and presentation is clear
and colourful, and I think it's a really good basis for giving an
intuition of different concepts in graph theory. 

Some nice possible extensions could be to allow for custom user graphs,
or provide a bit more detail on algorithms to compute these different
things like vertex covers.


May be it could help to give animations showing non-optimal solutions as
well (this can be seen for the interactive tasks but not for the problems
that did not have these)

An AMAZING job done. simple and fun to learn.
