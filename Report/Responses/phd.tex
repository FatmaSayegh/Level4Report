% The following responses are from Phd Candidates

The following responses are from Phd Candidates


Candidate 1

   fieldOfStudy: Computer Science/ Programming Languages

   familiarity: Had a brush in Computer Science Course

   stay: 15 - 30 min

   understand interact with the system: yes

   elements clear: 10/10

   features didn't use: keyboard shortcuts

   navigation: 10/10

   ux: I liked it! Very fluid. I liked the single-page-application feel, 
       rather than needing to reload different pages. 

   already aware topics: Graph Iso, Colouring

   which class appropriate: High School, Undergrad

   effective in explaining topics: 10/10

   which method better for understanding ani/task: both

   best explained: This is the first time I managed to understand 
                   tree width! I liked the way it was broken up into steps.

   not effective: I think all were effective, but it might have been nice to see animations 
                  and examples for graph colouring before doing the exercise.
   task participated: All
   features of app noticed (transform ani, dynamic explantion, user interaction, multiple ani, multiple task ): All
   extra feedback:
         """
         Do you have any extra feedback?
         * I really liked the animations, and especially the places where you hover 
           over vertices and the edges light up
         * When doing the exercise for isomorphisms, it might be nice to do a similar "edge hover", 
           because the exercise was more difficult without it
         * A lot of the time, "next animation / next task" cycled through rather than being greyed out, 
           so it wasn't easy to tell when the task / activity had finished
         * Maybe it'd be good to have an example of graph colouring before going into the exercise?
         * For tree width: are all graphs representable in that cellular structure?
         * Small typo: funcionality instead of functionality
         Overall: lovely app! Really liked it!
         """



Participant 2
   fieldOfStudy: Computer Science
   familiarity: Have done research in GT
   time: less than 5 mins
   elements of the screen: 7
   already aware topics: All
   which class appropriate: High School, Under grad
   effective teaching: 9/10
   ani/ interaction: both
   which topic best expln:  colouring
   not effective topicr:
      """ 
      Tree width. The application didn't full explain tree width, it just
      moved the graph a bit. I think tree width is just on a different level
      compared to the other graph theory concepts, and so it would need more
      detail dedicated to it. Perhaps it could be marked as an advanced topic.
      """
   user task participation: All
   features noticed: All

Participant 3
   fieldOfStudy: Computer Science
   familiarity: Sem course
   time: 5 <x< 10 mins
   elementsOfTheScreen: 8
   ux features not used :  shortcuts, home, about
   navigation: 7
   ux: "The visuals and animation were pleasant. Quite easy to navigate." 
   already aware: All
   learning effectivenes: 6
   method: Both
   best explained: Colouring, tree width.
   worst explained: "Perhaps isomorphism. Perhaps it a bit more instructive, if possible?"

participant 4
   fieldOfStudy: CompSci specially Graph Theory
   familiarity: research
   time: 5 - 10
   elments clear: 7
   features ux negative: shortcuts, buttons for next animation on the same topic
   already aware: All
   level of ed: high edu, undergrad
   learning effectiveness: 7
   best for understanding: tasks
   best explained:
      """
      I thought that max k cut was very well explained - the use of cutlines is rather nice.
      """
   worst explained:
      """
      Treewidth wasn't very clear, unfortunately - I think it wasn't clear that
      the treewidth of a graph is the smallest maximum bag size - 1 across all
      tree decompositions. So it seems like a graph can have multiple different
      treewidths, when it actually has multiple possible widths based on tree
      decompositions, then a treewidth that doesn't depend on which tree
      decomposition you use. It would have maybe been nice to include some
      detail on how treewidth is actually computed, though this is maybe a bit
      out of scope.  
      """

   extraFeedback:
      """ 
      Bit of an unfortunate typo on the about page... I assume you don't
      want to say that Sofiat's guidance was "detrimental" in building the app!

      On the whole, I really like the app! The design and presentation is clear
      and colourful, and I think it's a really good basis for giving an
      intuition of different concepts in graph theory. 

      Some nice possible
      extensions could be to allow for custom user graphs, 
      or provide a bit
      more detail on 
      algorithms to compute these different things like 
      vertex
      covers.
      """

participant 5:
   field: Comp Sci
   famil: Research
   time: 5 - 10
   elementsOfScreen: 7
   navigation: 7
   navigation not easy: placement Of navi buttons
   ux:
      """
      Very good.  Particularly liked the treewidth animation.
      """
   already aware: All
   class: Undergrad
   teach effectiveness: 7
   ani/task: Both
   best explained:
      "Treewidth due to identification of "blocks""
   not effective:
      "Graph isomorphism - maybe highlight which vertices in source are mapped
      to those in target"
   extraFeedback:
      """
      Very nice application!
      """

participant 6:
   field: Comp Sci
   famil: Research
   time: 5 to 10
   ele of scr: 9
   nav: 10
   ux:
      """
      The layout and experience is very good, it was intuitive and clear how to
      interact with the application.
      """
   aware: All
   class: Undergrad
   teach effectiveness: 9
   ani/task: Both
   best explained:
      """
      Graph isomorphism and vertex cover - the animations were great, but
      each it also had an interactive tasks"""
      worst explained:
      """
      This is very minor but the max cut animation could have added some way to
      highlight the different groups of vertices.
      """
   feedBack:
      """
      May be it could help to give animations showing non-optimal solutions as
      well (this can be seen for the interactive tasks but not for the problems
      that did not have these)
      """

participant 7:
   field: Comp Sci
   famil: Research
   time: 15 to 30
   Ux: "Huge fan of the simple design . User experience was very smooth and self explanatory."
   best explained:
      """ I loved all of them, if i was to pick I would say Tree Width , Max
      Cut and Graph colouring. Tree Width is a hard problem to understand , you
      have chosen an extraordinary way in visualising and explaining this
      problem, made it seem simple and easy to learn. Max Cut , I love the way
      the graph transforms then you add the cutlines it was creative. Graph
      colouring it is nice that you are able to colour the graph and see the
      edges get highlighted when a mistake has been made.  """
   
   worst explained:
      """
      Not at all , all of them are very effective and important . If i had to
      comment i would probably say even though i did say Tree Width was my
      favourite maybe you can try choosing better choice of words in the
      "explanation panel " such as "possible width decomposition" .
      """

participant 8:
   field: Comp Sci
   famil: Research
   time: 15 to 30
   Ux: Wow! impressive minimalist design, user experience was very smooth.
   best explained:
      """ Graph Isomorphism, and Tree Width. Graph Isomorphism plays an
      animation that transforms to another graph that "looks different" but you
      can see it is "isomorphic" as it is obviously derived from the Initial
      Graph, it is followed by a task which tests your knowledge and explains
      the" correct or incorrect" answer by just "looking" at it . Tree Width ,
      transforms to a "tree look alike " , the example you chose to visualise
      was very simple, and first step to learning it . Yes it is a
      "Visualisation web app" serves it purpose """

  feedback:
      """
      I have no words to say other than well done. As a Professor , I am
      amazed. continue evolving I believe this application will help many
      understand . Good Job.
      """
