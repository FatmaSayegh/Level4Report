% Analysis and Requirements

In this chapter I will cover the scope of the project, the criteria of selection of the
problems in graph theory, the thinking behind choosing the methods of
elucidation of the selected topics will be discussed. Finally the analysis
the requirements of the project are stated.


\section{Scope of The Project}
\label{requirements: scope}
Understanding a problem in mathematics is a necessary first step towards the solution.
Abstraction enables a student to obtain a formalized version of a
problem from a real life scenario present in the fields of science and engineering.

This idea has guided this project to focus solely on aiding a learner to
understand the problems in graph theory. The solution or suggesting an
algorithm to solve the problem, if required, is the second important step which
has been deliberately not touched upon to keep the scope of this project
clear and well defined.


\section{Criteria for Selection of Problems}
\label{section: selectionCriteria}
One of the most important criteria for selection of the problems for the
project was their importance and relevance in the field of graph
theory. There are several text books (\cite{Newman10}, 
\cite{KleinbergTardos06}) which discuss various theorems and
problems in the subject. A few problems occur frequently in them and the order of their inclusion in the text books is based
logically.  Building the concepts from the basics to advanced. Therefore the
problems included in the project should represent all levels of difficulty.

Since imagining a graph theory problem is largely a visual exercise, there are plenty of problems which could offer themselves as a subject of an
interesting visualization. The additional criteria therefore for filtering the
candidate problems was based on whether they could be elucidated in the form of a
simple and meaningful example, employed for animation and user
interaction. The simplicity of the example does not in any way imply
triviality of the problem. Indeed here the assumption is that a simple example
problem can make a student reach to the heart of the concept fairly quickly. From there on they can extrapolate what they learnt to more
complex examples.

A survey among my
peers in the field of software engineering and computer science was conducted to 
gather suggestions on the shortlisted topics and methods.
The data from the survey (see \autoref{links: initialConcept}) had a role in determining topics and the methods chosen.


\section{Methods of Elucidation}
Methods of elucidation and exposition played a prominent role in the selection of the problems
in the first place. It was decided that for this project, such methods can be
broadly classified as animations and user interactions or a combination of
both.


\subsection{Animation}

Humans tend to prefer to imagine even the most abstract concepts visually.
Computer animations have been increasingly adopted to elucidate complex
concepts in mathematics and science.  Animations will be used in this project to
make problems like Graph Isomorphism, Max Cut, and Tree Width.  For instance,
the example problem of Graph Isomorphism, was explained by morphing a graph to
change its shape to take on a completely unrecognisable appearance. Later in this report I will cover the topic of how it can be considered as a visual proof that the two graphs
in the scene are thus isomorphic.

\subsection{User Interaction}
Although animations go a long way in terms of having a user engaged and involved in
the learning process they only offer a one-sided, linear narrative. On the other hand,
user interaction with an animation not only makes the experience more
immersive, it also leads to a natural multiplicity of stories from a single
program. This happens as a human input results in a novel path from one state
to another in the program, not dissimilar to a video game.

\section{Technical Scope of The Project}
\label{requirements: techScope}

To achieve the outlined features the program needs data structures to hold
graphs and algorithms to visually and geometrically manipulate them. 

The program does not contain algorithms to solve
the listed problems. As the scope of the program was limited to the purposes of
visualization and not coding the algorithms which can solve instances of the
mentioned problems.  Therefore in this project, although the data type of
Graphs (Set of Vertices and Set of Edges) and the associated functions, are
quite general and can support operations of various kinds, care is taken that I
provide the solution to the visualization program before hand to give enough
information to the various animations and user interactions.


\section{Requirements}
\label{requirements: requirements}

The requirements for the application are categorized into priority
levels in accordance with \emph{MoSCoW} analysis. This prioritization of
requirements act a guidance for deciding which objectives are essential and
make the minimum viable product and which ones should come in the category of
extra desirable features. \cite{Hudaib2018}

The application must have the features mentioned in the \emph{Must
Have} section which is the minimum requirements necessary to have a functioning
application without any extra decorative or convenience features. It should have the additional
desirable features mentioned in the \emph{Should Have} section relating to
visual aesthetics, wider use and code extensibility.  Finally to precisely
define the boundries and discuss potential future development of the project there is section of \emph{Will Not Have
This Time}.

\subsection{Must Have}
\label{requirements: musthave}
The application must elucidate the following enumerated \emph{Classical Graph Theory
Problems} by employing user interactive animations of their respective examples.
\begin{enumerate}
\item Graph Isomorphism
\item Max Cut
\item Graph Coloring
\item Minimum Vertex Cover
\item Tree Width
\end{enumerate}

And to achieve this by programming the following items -

\begin{enumerate}
\item Data structures which represent vertices, edges and graphs.
\item Display the above entities as Scalar Vector Graphics on screen.
\item Translation and shape transformation of the graphs for animation.
\item Generation and handling of events triggered by user interaction with the elements in the animation for user interaction.
\item Display appropriate text in synchronization with the animations and user interaction.
\end{enumerate}

Doing so in a manner which fulfills the following subjective qualities -

\begin{enumerate}
\item Substantive learning impact
\item Ease of Use
\item Coherent Story telling
\end{enumerate}

And finally, evaluating the application with the help of the peers on
parameters which can be broadly classified into the following categories -

\begin{enumerate}
\item User Experience
\item Learning Impact
\item Quality of Elucidation
\end{enumerate}

\subsection{Should Have}
\label{requirements: shouldhave}
Although not essential for the basic utility and functioning of
the application, a few desirable features should be included for wider use and
additionally ease of contribution by others. 
\begin{enumerate}
\item Pleasing, appropriate aesthetics
\item Device compatibility --- the application must be usable on most common screen sizes.
\item Contribution friendliness, in the way of good code organization and documentation.
\end{enumerate}


\subsection{Will Not Have}
\label{requirements: willnothave}
For the reasons elaborated in \autoref{requirements: scope} and
\autoref{requirements: techScope} the project will not implement algoritmic
solution of the the graph theory problems.  Although it will have some
algorithms implemented to check if a given solution is correct specially in
user-interaction tasks.  The algorithmic solutions to the problems can be taken
up as future work for the application.
