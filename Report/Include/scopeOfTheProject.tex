% Analysis and Requirements

In this chapter, the scope of the project, the criterion of selection of the
problems in graph theory, the thinking behind choosing the methods of
elucidation of the selected topics will be discussed. Finally to conclude the analysis
the requirements of the project are stated.


\section{Scope of The Project}

Understanding a problem in mathematics is a necessary first step in trying to solve it.
It also enables a student to abstract out a formalized version of a
problem from a real life scenario present in the fields of science and engineering.

This idea has guided this project to be restricted to one which helps a learner to
understand the problems in graph theory. Whereas the solution or suggesting an
algorithm to solve the problem, if required, is the second important step which
has been deliberately not touched upon to keep the scope of this project
clear, precise and specific.


\section{Criteria for Selection of Problems}
\label{section: selectionCriteria}
One of the most important criteria for selection of the problems for the
project was based upon the importance of the topic in the field of graph
theory. There are several text books (see \cite{Newman10} and
\cite{KleinbergTardos06}), in graph theory, which discuss various theorems and
problems in the subject.  There are a few problems which occur commonly and
frequently in them. The order of their inclusion in the text books is based
logically.  Building the concepts from the basics to advanced. Therefore the
problems included in the project should represent all levels of difficulty.

Since imagining a graph theory problem is largely a visual exercise, there
was no dearth of problems which could offer themselves as a subject of an
interesting visualization. The additional criteria therefore for filtering the
candidate problems was based on whether they could be elucidated in the form of a
simple and meaningful example, employed for animation and user
interaction. The simplicity of the example doesn't in anyway imply
triviality of the problem. Indeed here the assumption is that a simple example
problem can make a student reach to the heart of the concept in it's generality
fairly quickly. From there on she can extrapolate the learning to more
complicated examples.

It is important to mention here that a survey among
peers in the field of software engineering and computer science was conducted to 
gather suggestions on the shortlisted topics and methods.
The data from the survey had a role in determining topics and the methods chosen.


\section{Methods of Elucidation}
As it has been discussed in the previous section that feasible methods of
elucidation/exposition played a prominent role in the selection of the problems
in the first place. It was decided that for this project, such methods can be
broadly classified as animations and user interactions or a combination of
both.


\subsection{Animation}
The visual medium has dimensions such as of color, position, shape and motion.
Generating them with a computer program, scenes can be created attracting the
users attention towards a particular aspect of it with the help of animations.
The scene in this project consists of a graph, which undergoes transformations
of color, shape and position to make a certain aspect of it emphasized for the
purpose of explaining a concept.  Animations are used in this project to make
problems like Graph Isomorphism, Max Cut, and Tree Width. For instance, the
example problem of Graph Isomorphism, was explained by morphing a graph to change
it's shape to acquire another radically different shape. It will be explained
later how it can be considered as a visual proof that the two graphs in the
scene are thus isomorphic.

\subsection{User Interaction}
Although animations go a long way in terms of having a user's mind involved in
the learning process they only offer a linear narrative. On the other hand,
user interaction with an animation not only makes the experience more
immersive, it also leads to a natural multiplicity of stories from a single
program. This happens as a human input results in a novel path from one state to
another in the program just like a video game.

\section{Technical Scope of The Project}

To achieve the above mentioned features
The program has data structures to hold graphs and
algorithms to visually and geometrically manipulate them. 

It's important to note here the program does not 
contain algorithms to solve the listed problems. As the
scope of the program was limited to the purposes of visualization and not
coding the algorithms which can solve instances of the mentioned problems.
Therefore in this project, although the data type of Graphs (Set of
Vertices and Set of Edges) and the associated functions, are quite general and
can support operations of various kinds, care is taken that I provide the
solution to the visualization program before hand to give enough information to
the various animations and user interactions.


\section{Requirements}
Based on synthesis of the above sections it is concluded that there must be a
web application written for the desktop browser with user interactive
animations. The requirements for the application are categorized into two
priority levels. The application must definitely have the features mentioned
in the \textbf{Must Have} section which is the minimum requirements necessary
to have a functioning application without any bells and whistles. It should
have the additional desirable features mentioned in the \textbf{Should Have}
section. 

\subsection{Must Have}
The application must elucidate the following \emph{Classical Graph Theory
Problems} by employing user interactive animations.
\begin{enumerate}
\item Graph Isomorphism
\item Max Cut
\item Graph Coloring
\item Minimum Vertex Cover
\item Tree Width
\end{enumerate}

And to achieve all this by programming the following items -

\begin{enumerate}
\item Data structures which represent vertices, edges and graphs.
\item Display the above entities as Scalar Vector Graphics on screen.
\item Translation and shape transformation of the graphs for animation.
\item Generation and handling of events triggered by user interaction with the elements in the animation for user interaction.
\item Display appropriate text in synchronization with the animations and user interaction.
\end{enumerate}

Doing so in a manner which fulfills the following subjective qualities -

\begin{enumerate}
\item Substantive learning impact
\item Ease of Use
\item Coherent Story telling
\end{enumerate}

And finally, evaluating the application with the help of the peers on
parameters which can be broadly classified into the following categories -

\begin{enumerate}
\item Quality of Elucidation
\item User Experience 
\item Learning Impact
\end{enumerate}

\subsection{Should Have}
The application should optionally have the following desirable features:
\begin{enumerate}
\item Pleasing Aesthetics, to be preferably verified by evaluation.
\item Mobile Friendliness, so that the application can be accessed anywhere.
\end{enumerate}
