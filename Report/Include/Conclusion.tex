% Conclusion
\section{Summary}

The application elucidates five graph theory problems by the help of animations
and user interactive tasks. It does this by animations and interaction with the
graph elements and input buttons. This app also has a dynamic explanation area
which contains text appropriate to the state of the lesson at a particular
time.

Clearly, the end goal of the project was to have some effectiveness in learning
outcomes. Therefore to elucidate the chosen topics, explanation strategies
were made carefully and animation friendly examples were chosen (and also designed). 

It was decided that the layout of the web pages will be divided equally for
display of animations and explanation text. The text on the \emph{explanation
panel} was \emph{state dependent}, so that it is synchronized with the state of
animation.

The program was written in Elm, which is a functional programming language.
Functional type system was used to define graphs, vertices and edges.  Vector
and linear algebra was used to draw and animate graphs on screen as SVG
elements. The program was divided into modules for the main program, graphs and
the various topics.

The web application was hosted on the internet and tested on screens of various
sizes. The application was subsequently modified to be adaptable to variable
screen sizes. This was done by using the width and height of the screen, and
adjusting the size of the elements on the screen and font-size accordingly. 

The application was evaluated by a survey among peers. This was useful in
collecting feedback to improve the application and remove mistakes. It also
informed about the limitations of the application specially lack of capability
to draw custom graphs.

The application's capability is currently limited to display, animation and
making simple queries. It can be extended for more advanced features like
adjacency lists and traversals. 


\section{Future Work}

The application can be extended according to different themes. It can be
extended either by enhancing the graph module to have more algorithmic
capabilities or enhancement of visualization methods or simply adding new
topics to the present list of topics.

\subsection{Enhancing Algorithmic Capabilities} 
Enhancing algorithmic capability of the application can be done in the
following directions.

\subsubsection{Explanation of Algorithmic Solutions to the Current Topics} 
Implementation of algorithm to find solution for some problems discussed in the
application. For example, a function to solve vertex cover using known
algorithms such as brute force is achievable with present capabilities of the
application.

\subsubsection{Traversals:} Implement traversal friendly graph type (with
adjacency lists) and algorithms to traverse it. This will help explaining
path traversal problems also their solutions.


\subsubsection{Shortest Path Algorithms:} Working of shortest path algorithms
working of Dijkstra's and A-star algorithms can be added to the topics. This
will require enhancement of algorithms in the graph module to execute the
mentioned algorithms.

\subsubsection{User Defined Graphs:} Implement functionality to allow user
instantiated graphs. Implement drawing facilities and a tool box to draw
vertices and edges.

\subsubsection{Compatibility with Graph Description Files:} Graph description files
should be able to be uploaded from or downloaded to the local device so that
the graphs generated by the application can be exported to other applications
as well. Or graph generated by the user somewhere else can be interpreted by
the application.

\subsection{Enhancement of Visualization}

\subsubsection{Clickable Words:} The definitions in the explanation panel
should be clickable and lead to a page in the application which has a more
elaborated mathematical (formal) definition of the term than is possible to fit
on the usual topic page.

\subsubsection{3D Animations:} Currently, the position and velocities are 2D
vectors (3D vectors with the $z$ always kept at zero). The animation can be
made more engaging by using 3D vectors and rotating them relative to all the
three axes.

\subsubsection{Field of View Capability:} 
For navigation inside large graphs, such as biological networks, zooming in
and out of graphs may come handy. This can be achieved by viewing a graph using
an abstract camera. Moving the camera with keyboard movements will change view
of the graph on the display. This capability can also be used to hop from one
graph motif to another or other explorations.

\subsubsection{Virtual Reality Capability:}
For a more enhanced 3D experience, libraries in Elm which render stereo images
for Virtual reality (VR) devices can be employed for exploration of larger
graphs. VR, if employed for right ends can be a game changer in the field of
online education.

\subsection{Addition of topics}
To add a topic to the application, a contributor has to define a new page for
the same and make use of core modules and integrate the module in the main
file. The link to the program can be found in \autoref{links: repository}.
Although addition of advanced topics may demand enhancing core functionality as
well.

\section{Reflections}

The project was an important learning experience in design, implementation and
evaluation of a project of the size. Here are some insights into this experience.

\subsection{On Using Functional Programming} 
\label{reflection: functional}

A web app today is one of the primary means of human computer interaction
(HCI). To model something as complex as a HCI a programming paradigm which can
model this complexity efficiently is needed. 

An imperative programming language, however \emph{high-level} it might be, is
much closer to the Von Nuemann machine in terms of how a computer program is
reasoned about by the programmer. There is a tape and it has to be read one
step at a time. The programmer implements most of the logic in terms of control
mechanisms and less in terms of logic itself. Therefore the program becomes far
more complex than the actual complexity it has to represent. 

Functional programming on the other hand was much more efficient in modeling
the domain. It implemented logic more in terms of logic and less in terms of
control structures. Such that while reading the program one mostly saw logic
and not boiler-plate code.

\subsection{On Project Management}
For a project of this size the adage \emph{well begun is half done} is true.
The concluding days of the project should be preferably reserved for
evaluation, and feedback and subsequent changes.

During the development of the application it was observed that after acquiring
basic drawing and animation techniques for implementing the first topic, the
pace of development accelerated with the help of reusable data structures and
animation functions.

\subsection{On Learning Effectiveness}

Graph Theory and mathematics in general are studied using formal language. It
is left to the student for visualizing such concepts abstractly most of the
times. Therefore computer visualization can be used as an aid in learning
especially for the students who are new to the field.

Visualization of graph topics is limited to certain examples which lend
themselves well for animation. Not all instances of graphs can do this.
Therefore choosing the right examples was a crucial decision for each topic.

Though, it should be mentioned that oversimplification of the topic sometimes
hides more than it reveals. Using a particular example of a topic has the
potential to harm it's generality. Therefore, one should be careful and give
emphasis to formal methods in mathematics equally if not more.

\section{Concluding Thoughts}

Web applications have become today the primary interface between humans and
computation and communication. Front end programming is therefore responsible
for the interface between most humans and computer services. Therefore a web
application can be an convenient tool for teaching visual concepts like maths
and geometry.

Teaching, is hard; it should be harder still when it is done in absence and let
a program do it for you. And though a lesson should be simple, it should not
be simpler than it ought to be. This can give rise to complexity and non-linear
stories. A program (and the programming paradigm) must be robust enough to
handle such complexity. 

While designing and implementing this project the user's intuition was my
primary client. It was stepping into her shoes, that directed me to choose the
examples, design the functionalities and decide on the layout among other
things.
