\hypertarget{terms-and-definitions}{%
\section{Terms and Definitions}\label{terms-and-definitions}}
The following are the basic terms and definitions which are
used in Graph theory.
\hypertarget{some-basic-definitions}{%
\subsection{Some basic definitions:}\label{some-basic-definitions}}

\begin{itemize}
\item
  Vertices: A graph has vertices. Which can be understood as points.
\item
  Edge: An edge is an unordered pair of vertices.
\item
  V(G) is the vertex set of graph G.
\item
  E(G) is the edge set of graph G.
\item
  Degree of a vertex: deg(v). Number of edges connected to a vertex.
\item
  isolated vertex v: deg(v) = 0
\item
  End vertex w: deg(w) = 1
\end{itemize}

\hypertarget{some-basic-facts}{%
\subsection{Some basic facts}\label{some-basic-facts}}

\hypertarget{handshaking-lemma}{%
\subsubsection{Handshaking Lemma}\label{handshaking-lemma}}

In any graph the sum of all the vertex-degree is an even number - in
fact, twice the number of edges, since each edge contributes exactly 2
to the sum. Total summation of is edes*2

\hypertarget{adjacency}{%
\subsubsection{Adjacency}\label{adjacency}}

Two vertices are adjacent if they have an edge between them. Two edges
are adjacent if they have a vertex in common.

\hypertarget{what-graphs-are-not-about}{%
\subsubsection{What graphs are not
about}\label{what-graphs-are-not-about}}

Metrical properties: length and shape of the edges. (Irrelevant in graph
theory). (so if an edge is smaller, larger etc doesn't matter what
matters if the edges between two vertices are connected or not )

\hypertarget{some-more-definitions}{%
\subsubsection{Some more definitions}\label{some-more-definitions}}

\hypertarget{simple-graphs}{%
\paragraph{Simple graphs:}\label{simple-graphs}}

Graphs with no loops and multiple edges.

\hypertarget{general-graph}{%
\paragraph{General graph:}\label{general-graph}}

Loops and multiple edges are allowed.

\hypertarget{digraphs}{%
\paragraph{Digraphs:}\label{digraphs}}

Directed graphs. When edges have arrows.

\hypertarget{walk}{%
\paragraph{Walk:}\label{walk}}

A way of getting from one vertex to another, and consists of a sequence
of edges. P -\textgreater{} Q -\textgreater{} R is a walk of length 2

\hypertarget{path}{%
\paragraph{Path:}\label{path}}

A walk in which no vertex appears more than once.

\hypertarget{cycle}{%
\paragraph{Cycle}\label{cycle}}

A path like this: Q -\textgreater{} S -\textgreater{} T -\textgreater{}
Q is called a cycle.

\hypertarget{subgraph}{%
\paragraph{Subgraph}\label{subgraph}}

G - e is the graph obtained from G by deleting the edge e.

\hypertarget{adjecency-matrix}{%
\paragraph{Adjecency Matrix}\label{adjecency-matrix}}

If G is a graph with vertices labelled \{1,2,..n\}, its adjacency matrix
A is n * n matrix whose ijth entry is the number of edges joining vertex
i and vertex j.

\hypertarget{null-graph}{%
\paragraph{Null Graph}\label{null-graph}}

Edge set is empty. All vertices are isolated.

\hypertarget{complete-graph}{%
\paragraph{Complete Graph}\label{complete-graph}}

A simple graph in which each pair of distinct vertices are adjacent is
complete. Complete Graph of n vertices are denoted as Kn. They have
n(n-1)/2 edges.

\hypertarget{cycle-graphs}{%
\paragraph{Cycle Graphs}\label{cycle-graphs}}

A connected graph which is regular of degree 2. A cycle graph of n
vertices is denoted by Cn.

\hypertarget{path-graph}{%
\paragraph{Path Graph}\label{path-graph}}

A graph obtained from Cn by removing an edge. Denoted by Pn.

\hypertarget{wheel}{%
\paragraph{Wheel}\label{wheel}}

The graph obtained from Cn-1 by joining each vertex to a new vertex v is
wheel on n vertices, denoted by Wn.

\hypertarget{regular-graph}{%
\paragraph{Regular Graph}\label{regular-graph}}

Each vertex has the same degree. With that degree r, the graph is a
regular of degree r of r-regular.

\hypertarget{bipartite-graph}{%
\paragraph{Bipartite Graph}\label{bipartite-graph}}

If the vertex set of a graph G can be split into two disjoint sets A and
B such that each edge of G joins a vertex of A and a vertex of B, then G
is a bipartite graph. (like when we divide it into two sets anything in
a will connect with set b not another set of a)

\hypertarget{hamiltonian-graphs}{%
\paragraph{Hamiltonian graphs}\label{hamiltonian-graphs}}

Graphs containing walks that include every vertex exactly once, ending
at initial vertex (so we should start and end in same point without
repeating vertices not all vertices have to be used).
