\hypertarget{topic-explanation}{%
\section{Topic Explanation}\label{topic-explanation}}

\hypertarget{some-classical-graph-problems}{%
\subsection{Some Classical Graph Problems:}\label{some-classical-graph-problems}}

\begin{itemize}
\tightlist
\item
  Graph Isomorphism
\item
  Hamiltonian Cycle
\item
  Graph Colouring
\item
  Indentification of Independent Set
\item
  Identification of Cliques in a Graph
\item
  Finding Minimum spanning tree in Graph
\item
  Finding of Vertex cover
\item
  Max k-Cut
\item
  Tree Width
\end{itemize}

\hypertarget{graph-isomorphism}{%
\subsection{Graph Isomorphism:}\label{graph-isomorphism}}

Two graphs G1 and G2 are isomorphic if there is a one-one correspondence
between the vertices of G1 and G2 such that the number of edges between
any two vertices in G1 is equal to the number of edges joining the
corresponding vertices of G2. Here the graphs may appear to be different
in appearance and the labeling of the nodes and edges. But the way one
vertex is connected to another in one graph is same to another.
Therefore given two graphs, detecting if the graphs are Isomorphic is a
problem to solve. One way to explain this would be to manipulate the
position of vertices and edges to be appear same as it's isomorphic
counterpart. We want to show what isomorphism is.

Visualization: - Display two graphs which are visibly (topologically)
different but isomorphic.

User Interaction: - Given a initial graph which the user can manipulate
by draging its vertices to make it look equivalent to it's isomorphic
equivalent. - Position of vertices and edges in one of the graphs can be
moved in an animation to make it look visually the same as its
isomorphic counterpart. Graph drawing algorithms should be employed
here.

\hypertarget{hamiltonian-graphs}{%
\subsection{Hamiltonian graphs:}\label{hamiltonian-graphs}}

Graphs containing walks (moving from one edge to another) that include
every vertex exactly once, ending at initial vertex. (so we should start
and end in same point without repeating vertices and cover all the
vertices).

Visualization: - Explaination can be given by giving an example graph
with a hamiltonian cyle highlighted on it. - We can also give negative
examples of paths which are not hamiltonian. - There can also be an
Exercise for the user in which she is given a graph and and asked to
mark the edges which would make a hamiltonian cycle. - The program
should check if that is a hamiltonian cylce by checking if all the
related constraints defined in the definition of hamiltonian cycle are
satisfied which are if all the vertices are visited and also that no
vertex has been visited twice.

\hypertarget{graph-clouring}{%
\subsection{Graph Clouring}\label{graph-clouring}}

It is an optimization problem, where the objective is to assign to
vertices of a graph a colour such that no two adjacent vertices have the
same colour, while keeping the number of colours employed to a minimum.


Visualization: A properly coloured graph, which satisfies all the
constraints can be displayed. - Integer Linear Programing can be
explained here as well.


User Interaction: User is allowed to colour vertices. The program
will warn the user if adjacent nodes are coloured the same. Will be
like solving sudoku.

\hypertarget{identification-of-maximum-independent-set}{%
\subsection{Identification of Maximum Independent
Set}\label{identification-of-maximum-independent-set}}

An Independent Set of a graph is a set of vertices such that no two
vertices in that set are adjacent to each other. A Maximum Independent
set is an Inpendent set with the largest possible number of vertices.

Visualization: - Show a graph with members of an Independent Set
coloured differently from the rest of the vertices. - Show the same
graph with maximum independent set.

\hypertarget{identification-of-cliques-in-a-graph}{%
\subsection{Identification of Cliques in a
graph}\label{identification-of-cliques-in-a-graph}}

A clique is a set of vertices of a graph such that all the vertices are
connected to each other. This set is defined in such a way that there is
no other vertex in the graph which can be added to the set, while
preserving the property that all the vertices are connected to every
other. Visualization: - In a graph, a clique can be highlighted by
colouring the vertices and the edges involed in the clique.

\hypertarget{finding-minimum-spanning-tree-in-graph}{%
\subsection{Finding Minimum spanning tree in
Graph}\label{finding-minimum-spanning-tree-in-graph}}

In a connected graph, minimum spanning tree is a subset of edges such
that, all the vertices are connected. This should be a tree without any
cycles and the summation of weights should be minimum if there are more
than one spanning trees present in the graph.

Visualization: - In a succession the following can be done - Show what a
connected graph is. (Define what a connected graph is.) - Draw spanning
tree on the graph. (Define what a spanning tree is.) - Draw a minimum
spannig tree. (Show that the constraints of the definition are
satisfied.)

\hypertarget{minimum-vertex-cover}{%
\subsection{Minimum Vertex Cover}\label{minimum-vertex-cover}}

Minimum Vertex cover of a graph is the minimum amount of vertices such
that, all the edges in the graph must have one of such vertices as at
least one of their endpoints.

Visualization: Animation: - Show how only a few vertices can cover all
the edges, and such vertices are the vertex cover of the graph. - Show
all the constraints of the definition are satisfied.

Interactive: - The user chooses an vertex and all the edges incident on
it light up. - Keeps choosing a vertex until all the edges light up.

\hypertarget{max-k-cut}{%
\subsection{Max k-Cut}\label{max-k-cut}}

A maximum cut, is partioning the vertices of a graph in two groups such
that the number of edges between these two groups is maximum. In a
weighted graph, where the edges are weighted, the weights of the edges
are also taken into consideration. A maximum k-cut, is generalised
version of maximum cut, where the graph is partitioned into k subsets,
such that the number of edges between these groups is maximised.

Visualization:

\begin{enumerate}
\def\labelenumi{\arabic{enumi}.}
\item
  2-cut Animation:

  \begin{itemize}
  \tightlist
  \item
    Making a 2-cut by a curved line on a simple graph to show how
    maximum number of edges have been cut between the two groups of
    vertices. or classifying the vertices of the two sets by depicting
    them in two different graph.
  \item
    Then clustering them and seperating the two groups by a distance (of
    course without breaking any edges) to show the flow of edges between
    the two groups.
  \end{itemize}
\item
  Bipartite Graph (as a trivial example):

  \begin{itemize}
  \tightlist
  \item
    Showing that a bipartite graph when it is recognised as a bipartite
    graph is already a trivial 2-cut graph.
  \end{itemize}
\item
  User Interaction:

  \begin{itemize}
  \tightlist
  \item
    Let the user choose the vertices he wants to be in set A. The rest
    of vertices will auto-matically will be assigned to set B. The
    program will show the number of edges which flow between the two
    sets.
  \end{itemize}
\item
  Animation for k-cut:

  \begin{itemize}
  \tightlist
  \item
    Making a 3-cut by two curved lines on a simple graph to show how
    maximum number of edges have been cut by the two lines.
  \item
    Making a 2-cut by a line on a weighted graph, where the cut is made
    keeping the weights of the edges into consideration.
  \end{itemize}
\item
  \begin{itemize}
  \tightlist
  \item
    Ideas on constructing a graph such that max cut (2-cut) is known
    before hand.
  \item
    Take two sets of vertices, A and B, initialy with zero edges in the
    graph. Whenever an edge is made between the vertices of A, then two
    edges is made between vertices of set A and B.
  \end{itemize}
\end{enumerate}

\hypertarget{tree-width}{%
\subsection{Tree Width}\label{tree-width}}

We will explain in two parts. First we will define what a tree
decomposition of a graph is. Then we will define tree width of the
graph.

Formally, a tree decomposition of G = (V,E) consists of a tree T and a
subset Vt \textless= V associated with each node t \textless- T. We will
call the subsets Vt pieces of tree decomposition. T and Vt must satisfy:
- (Node coverage) Every node of G belongs to at least one piece Vt. -
(Edge coverage) For every edge e of G, there is some piece Vt containing
both ends of e. - (Coherence) Let t1, t2 and t3 be three nodes of T such
that t2 lies on the path from t1 to t3.

All graphs have tree decompositions. A trivial tree decomposition of any
graph has just one node with all the vertices of the graph residing in
it. It will satisfy all the conditions mentioned in the definition of
tree decomposition. But this trivial decomposition will not be useful as
it will not have seperable properties of a tree.

Therefore we must look for tree decompositions which have small pieces.
Tree width is defined as the size of the biggest Vt - 1.

Therefore tree width of G is the minimum width of any tree decomposition
of G.
