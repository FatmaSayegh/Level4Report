% Introduction

\section{Motivation}
There are various kinds of phenomenon in nature and concepts in computer science which can
be modeled as graphs. Graphs can be thought of as points (called vertices or
nodes) related to each other by edges. The points can be humans in a social
network, genes in a gene expression network, proteins and metabolites
interacting with each other in a cell, various living species in an ecology or
a food-web. The nature of relations between such objects can be causation,
interdependence, interaction, etc.  Neural Networks, both natural and
artificial are graphs too.  There is a new field emerging in computer science
and statistical mechanics called \emph{Complexity}, which promises to enlighten
us about phenomenon such as life, ecology, intelligence, society and economy.
Graphs stand as one of the fundamental tool to model and study such complexity.
See Chapter 1 of \cite{Gros2015}.

Reducing a real world problem to a graph may amount to loosing details but as
graph theory has developed a number of tools for analysis and insight, these
tools become available immediately for a well formed inquiry. Therefore once a
concept in Graph theory is learned then just like other concepts in Mathematics
such as integers or vectors etc, it can be applied in a variety of fields of
study.

Although the most dependable way to study Graph Theory should be mathematical
formalism to state the terms, definitions and theorems; visualization of such
concepts can act as a first stepping stone for the uninitiated. It can also act
as an aid for a practitioner to enrich his understanding or to view the same
concept the same concept in a different light.

Using a particular example to define a topic in mathematics may diminish it's
generality. But a particular instance of a topic can give a concrete confidence
to a student about his understanding of the topic in consideration. It is
expected that a concrete example can be extrapolated by a student's mind for a
variety of situations and also a general understanding.

\section{Aim}
The aim of this project is to create computer visualization of classical graph
theory problems. To that end graph theory problems which are important in the
field should be shortlisted along with searching and designing of simple
examples and user stories to elucidate the definition of such problems.
The method of elucidation should be animation, interactive tasks and textual
explanation on a modern front end web application. To achieve this, a program
should be written to represent graphs and conduct basic operations on them;
with functionality to spawn such graphs visually on screen and
animate them using basic kinematics.

Finally, the advice of peers in the field should be sought to evaluate the
application in terms of usability and learning impact.

\subsection{Link to the Application}
For the reader's reference the application can be visited following this
URL: \\
\href{https://visualise-graph-problems-with-me.netlify.app/} {https://visualise-graph-problems-with-me.netlify.app/}.
%This project will take a route which is not taken often in terms of the
%programming paradigm it will employ. The program will be written in Elm, a
%functional programming language which makes the logic and implementation for
%the size of the project readable, reasonable and manageable.


% Elaborate about what are problems in Graph Theory.

% History of Graph Theory Problems?

% Survey of Graph Theory Problems?

% Survery of Applications of Graph Theory Problems?

% 


%\begin{enumerate}
%\item Graph Theory What?
%\item Why? Applications?
%\item How? History?
%\end{enumerate}
%
%\section{Motivation}
