% Introduction
There are various kinds of phenomenon in nature and computer science which can
be modelled as graphs. Graphs can be thought of as objects related to each
other. These objects can be humans in a social netwrok, genes in a gene
network, proteins and metabolites interacting with each other in a cell,
various living species in an ecology or a food-web. The nature of relations
between such objects can be causation, interdependence, interaction, etc.
Neural Networks, both natural and artificial are graphs too. There is a new
field emerging in computer science and statistical mechanics called
\emph{Complexity}, which promises to enlighten us about phenomenon such as
life, ecology, intelligence, society and economy. Graphs stand as one of the
fundamental tool to model and study such complexity.

% Elaborate about what are problems in Graph Theory.

Although the most dependeble way to study Graph Theory should be mathematical
formalism to state the terms, definitions and theorems. Visualization of such
terms can act as easy first stepping stone for the uninitiated. It can also act
as an aid for a practitioner to enrich his understanding or to view the same
concept the same concept in a different light.

It's true that using a particular example to define a topic in mathematics may
diminishes it's generality. But a particular instance of a topic can
give a concrete confidence to a student about his understanding of the
topic in consideration. It is expected that a concrete example can be
extrapolated by a student's mind for a variety of situations and also a
general understanding.




%\begin{enumerate}
%\item Graph Theory What?
%\item Why? Applications?
%\item How? History?
%\end{enumerate}
%
%\section{Motivation}
