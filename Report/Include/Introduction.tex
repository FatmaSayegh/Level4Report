% Introduction

% Motivation
There are various kinds of phenomenon in nature and computer science which can
be modelled as graphs. Graphs can be thought of as points (called vertices or
nodes) related to each other by edges. The points can be humans in a social
netwrok, genes in a gene expression network, proteins and metabolites
interacting with each other in a cell, various living species in an ecology or
a food-web. The nature of relations between such objects can be causation,
interdependence, interaction, etc.  Neural Networks, both natural and
artificial are graphs too.  There is a new field emerging in computer science
and statistical mechanics called \emph{Complexity}, which promises to enlighten
us about phenomenon such as life, ecology, intelligence, society and economy.
Graphs stand as one of the fundamental tool to model and study such complexity.
See Chapter 1 of \cite{Gros2015}.

Reducing a real world problem to a graph may amount to loosing details but as
graph theory has developed a number of tools for analysis and insight these
tools become available immideately for a well formed inquiry. Therefore once a
concept in Graph theory is learned then just like other species in Mathematics
such as integers or vectors etc, they can be applied in a variety of fields of
study.


Although the most dependeble way to study Graph Theory should be mathematical
formalism to state the terms, definitions and theorems; visualization of such
concepts can act as a first stepping stone for the uninitiated. It can also act
as an aid for a practitioner to enrich his understanding or to view the same
concept the same concept in a different light.

It's true that using a particular example to define a topic in mathematics may
diminish it's generality. But a particular instance of a topic can give a
concrete confidence to a student about his understanding of the topic in
consideration. It is expected that a concrete example can be extrapolated by a
student's mind for a variety of situations and also a general understanding.

% Elaborate about what are problems in Graph Theory.

% History of Graph Theory Problems?

% Survey of Graph Theory Problems?

% Survery of Applications of Graph Theory Problems?

% 


%\begin{enumerate}
%\item Graph Theory What?
%\item Why? Applications?
%\item How? History?
%\end{enumerate}
%
%\section{Motivation}
