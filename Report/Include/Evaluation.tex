% Evaluation

\section{Evaluation}

An application such as this one, which aims to explain mathematical or
scientific concepts based on primarily animations and tasks must be evaluated
on two major aspects . The first aspect being user interaction and user
experience (UI/UX). The second one being the learning effectiveness.


The evaluation of the application was done on the basis of feedback from peers
mainly in the field of computer science and mathematics. There were two surveys
conducted, the first was unsupervised and anonymous. The second survey was
supervised in which the participants were inteviewed while they were
experiencing the application.

\subsection{Unsupervised Survey}
The unsupervised survey was conducted among peers mainly in the field of
computer science, mathematics and education. The participants were given the
URL of the application and an online feedback form. The useres were asked to
visit the application and and answer a questionaire.

Although the survey was anonymous, i.e. the names of the participants remained
unknown, but details such as their academic qualification and familiarity with
the subject of graph theory were obtained to put the responses in context.

The participants consisted of eleven PhDs, eight of which belonged to the field
of computer science. There were four participants with master's degree all of
whom belonged to computer science field. There were nine undergraduates mainly
from the field of computer science and engineering.

Among the eight PhDs in computer science, six mentioned that they have done
research in the field of graph theory, while the remaining two mentioned that
they had a course either exclusively on the subject or a wider field.
Therefore the feedback on the application brought a substantive quantity and
quality of opinions, criticism and even some praise.

The feedback ranged from opinions on UI/UX to 
opinions on effectiveness of a particular topic.

\subsection{Supervised Survey}
The supervised survey consisted of the same questions the as unsupervised
survey . Though, unlike the former survey the web application was visited by
the participants by sharing their computer screen. The participants unlike the
unsupervised survey belonged to a variety of education backgrounds like
humanities, sciences and other fields. Therefore the participants were
frequently assisted while understanding the animations and attempting the
quizes and tasks.

This gave the opportunity to see closely how the users navigate the app and
which aspects of the UI and the learning material they find confusing.

The Unsupervised Survey indicated that this application may be usable by high
school and mid school students. Therefore a supervised survey with young
participants was also done to explore the reach of the application's
effectiveness in that age group. This was done after appropriate consent from
their parents.

\subsection{Evaluation of UI/UX}
The survey was divided into two sections. The first part dealt with the user interface
and experience aspect of the application. There was both praise and criticism
from the participants. Both were taken into consideration and analysed
in this section.



\subsubsection{Praise for UI/UX:}
Positive feedback for the UI/UX include the following comments:

\begin{quote}
\emph{``Very fluid. I like the single-page-application feel, rather than
       needing to reload different pages.''}
\end{quote}



\begin{quote}
\emph{``The visuals and animation were pleasant. Quite easy to navigate.''}
\end{quote}



\begin{quote}
\emph{``.. it was intuitive and clear how to
      interact with the application.''}
\end{quote}


The first comment in the praise about the application being a single page
application (SPA) underlines an important aspect of the UX of the application.
It removes distractions in the users learning experience. This is unlike
ordinary applications which while loading a new page briefly shows a white
screen even on fast connections.


\subsubsection{Criticism of UI/UX:}
There was constructive criticism of the UI/UX of the application as well.
Most of these criticisms were actionable while one criticism, given the
constraints put by the requirement of the topics was not.


\begin{quote}
\emph{``One thing I found a bit frustrating was the varying levels and styles of
interactivity between topics, such as clicking the cutline for max cut .. ''}
\end{quote}

\begin{quote}
\emph{``Maybe the topic words could be clickable to make navigation easier.''}
\end{quote}

\begin{quote}
\emph{``There are occasional spelling errors that
   could be corrected ..'' }
\end{quote}

The first criticism in the list above points out that the different topics had
different styles for interactivity.  Although, it is agreed that all the pages
should have consistency in the number and nature of interactable buttons
but the different demands of each topic made it difficult.  For example the max
cut page has a cut-line button and others don't. This was unavoidable.
However, it is understood well that a variation in button layout and
functionality is potentially quite distracting. Perhaps, in a future iteration
of the application a drawer of tools which slides out and in can be made for
each page, though it may have it's own drawbacks in terms of visibility and
accessibility.

However most criticism of the UI/UX like clickable names on the navigation bar
were actionable.  For example, the users tend to click the workds on the
navigation bar. Change in the implementation was made for the same. For
spelling errors in the explanation panel, they were combed out and corrected.


\subsection{Evaluation of Learning Effectiveness}
In general, from the feedback of the respondents, it could be gathered
that the application in learning effectiveness.

\subsubsection{Most Effective Topic:}
To obtain what the respondents think about the learning effectiveness of the
application they were asked which topic in the app did they find most effective
for learning. The answers ranged over all the topics. This indiacates that every
topic had something to impress the respondents in terms of learning efficacy.


The respondents who are familiar with the field of graph theory mentioned
\emph{Tree Width}, and \emph{Max k Cut} the most. Although they mentioned rest
of the topics at least once. The two topics made to the top spot as they are
two most difficult topics to understand.

They were also asked about the least effective topic and the reason of their choice. 
This revealed some of the weaknesses in topic explanation. Again Tree width
faced the most criticism for lack of clarity in the explanation panel. 

\subsubsection{Graph Isomorphism:}
Graph Isomorphism was mentioned in the feedback positively for having both an
animation and a quiz. The animation was appreciated for having the
functionality in which, when a user hovers over vertices and the vertices and
the edges light up. 

However, there was a suggestion for the quiz 
that it would be nice if do a similar ``hovering functionality'' over the
vertices because the quiz was more difficult without it.

\subsubsection{Max k Cut:}
This was one of the most mentioned topic in the survey in a positive way,
perhaps because the difficulty in understanding it's definition. The cut-line
functionality has been praised by multiple respondents. Here are a few
comments on that from the respondents.

\begin{quote}
\emph{``Thought that max k cut was very well explained - the use of cutlines is
rather nice.'}
\end{quote}

\begin{quote}
\emph{``I love the way the graph transforms then you add the cutlines it was
creative.''}
\end{quote}

\subsubsection{Graph Coloring:}
The topic received mentions for the user interactive task. Perhaps, user
interaction is much favored way of learning than a passive animation.
It was most praised by the respondents who were not much familiar with
the field of Graph Theory. It should be so because it is a relatively easy
topic to understand compared to the other topics.

\subsubsection{Vertex Cover:}
Just like graph coloring, vertex cover found mention in the most effective topic for 
containing a user interactive task.

\subsubsection{Tree Width:}
Tree width was in the special focus of the respondents who are researchers in
the field.  Tree width received both praise and criticism. It received praise
for being at a different level of difficulty from the rest of the topics.
\begin{quote}
\emph{``
Tree Width is a hard problem to understand , you
have chosen an extraordinary way in visualising and explaining this
problem, made it seem simple and easy to learn.''}
\end{quote}

Identification of the pieces and decomposition was pointed out by various
respondents who are specialist in the field as significant in understanding the
topic. For example the following comments point the same out.

\begin{quote}
\emph{``
Tree width due to identification of blocks
''}
\end{quote}
\begin{quote}
\emph{``
I liked the pieces and decomposition .
I finally managed to start understanding Tree Width!''}
\end{quote}

Credit should be given to the book \cite{KleinbergTardos06} which lent a very
clear example of tree width which was adopted for the animation for this topic.

Tree width also received some constructive \textbf{criticism}. There was
ambiguity in the explanation of the possible tree decompositions of a graph and
possible tree widths corresponding to such decompositions. Here the emphasis
should be given to the word \emph{possible} which was not put at the appropriate
place in the explanation of the topic in the app. This was pointed out
by two respondents who are experts in the field. 

\begin{quote}
\emph{``
Treewidth wasn't very clear, unfortunately - I think it wasn't clear that the
treewidth of a graph is the smallest maximum bag size - 1 across all tree
decompositions. So it seems like a graph can have multiple different
treewidths, when it actually has multiple possible widths based on tree
decompositions, then a treewidth that doesn't depend on which tree
decomposition you use....''}
\end{quote}
\begin{quote}
\emph{``
Tree Width was my favourite maybe
you can try choosing better choice of words in the `explanation panel'  such as
`possible width decomposition'.''}
\end{quote}

Following their advice, the text in the animation panel was changed so that
the concept of tree width is clearly and correctly understood by the users.


\subsection{Extra Feedback from the Respondents}
The application was commended for being clear and colorful and being
potentially extensible to other concepts of graph theory by most participants.
There were a few suggestions for extending the functionality such as:
\begin{enumerate}
\item Custom user graphs.
\item Detail on algorithms to compute these problems like vertex covers.
\item Animations showing non-optimal solutions.
\end{enumerate}


\subsection{Limitations}

The application has several fundamental limitaions some of which are pointed
out in the survey by the participants. Others are have been deduced
independently.

\subsubsection{Absence of Algorithmic Solutions:}
Since the objective of the application was to define and state the problems
clearly and not to solve them algorithmically, the explanation of algorithms to
solve the problems have been steered clear of. This limitiation in the scope of
the project remains a limitation of the application.

\subsubsection{Easy Examples:}
The examples used to explain the topics are simple. For instance, the exmaple
chosen to explain the max 3 cut problem was a tripartite graph.  It's solution
is trivial. These examples were chosen to clarify the definitions of the
problems, even though complicated examples could have given an uninitiated user
a flavor of complexity and hardness of the problems.

\subsubsection{Absence of Custom User Graphs Functionality:}
There are web applications on the internet, which allow the user to instantiate
their own graphs, with user instantiated vertices and edges. Although such
functionality has been avoided as the application's objective is to clarify the
definition of the problems. For this purpose in the application there are only
concrete examples which lend themselves to easy explanation of the problem at
the hand. However if in the future, algorithmic solutions to such problems, is
also included, then functionality of defining custom graphs can be added.

\section{Testing}
Automation of testing for this project was done by using a mix of Property
based Testing and Unit Testing. Both paradigms of testing can be implemented
for Elm projects using the library called \emph{elm-test} and a cli-tool which
goes by the same name. Writing test files in Elm is the same as writing any
other program. The test functions can generate input test cases (specially in
the case of property based testing) and use boolean predicates to generate the
output in terms of $Pass$ or $Fail$.

\subsection{Property Based Testing}
Testing of components of Graph module was done using Property Based Testing (PBT).
This paradigm of testing was introduced for the first time for Haskell (a
functional programming language). It checks if the output of a function
satisfies certain prescribed properties. These properties are specified as
boolean predicates.

For example, a function which constructs a fully connected graph (a graph in
which all the vertices are connected to each other by one edge) with $n$
vertices, will form $n \times (n - 1) / 2$ edges. The number of edges of the output
graph are matched up against this mathematical reality. The function passed
the test against a vast number of auto generated inputs.

Similarly, a cyclical graph i.e one in which the vertices are connected in a
pattern such as this: A -- B -- C -- D -- A, with n vertices will have the
number of edges equal to $n$. A function which constructs such a graph can be
property based tested by comparing the number of edges it has with the number
of vertices, which in this case should be equal.

\subsection{Unit Testing}
Unit testing was done on for the functionality of graph animations and
functions reponsible for navigation in the application against possible inputs
and referece outputs. This test was written using the \emph{elm-test} library
and run via a \emph{cli-tool} which goes by the same name.

\subsection{Manual System Testing}
The system was continuously tested manually during the development of the
project. Regardless, towards the end of the development a planned scheme for
interacting with the system for the black box testing of the application was
executed to search for unexpected behavior. See \autoref{appendix:
manualTesting} for the details of the test and results. The application's
behaviour against all the tests was as expected.
