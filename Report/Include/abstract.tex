Whereas, formalism in mathematics gives the subject a structure and a level of
abstraction which makes it applicable to the most general of the situations.
Visualization of a concept on the other hand, restricts the theory to a
particular example but it makes one to see a problem in a more concrete
pattern. It may also allow the learner to see the same problem in a new light. The
learner can then apply or extrapolate the learning to other instances of the problem in
general. 

In this project an application to elucidate some classical problems in Graph
Theory by the way of visualization is developed. The methods of visualization
are animations and user interaction with such animations. These methods, it is
believed, can help young students and self-learners as to get the first brush
with the subject of Graph theory.

The details and experience of analysis, design, development, testing and
evaluation of the application is discussed in this report.

The web application can be reached following
\href{https://visualise-graph-problems-with-me.netlify.app/} {this} 
\href{https://visualise-graph-problems-with-me.netlify.app/} {URL:} \\
\href{https://visualise-graph-problems-with-me.netlify.app/} {https://visualise-graph-problems-with-me.netlify.app/}
.
